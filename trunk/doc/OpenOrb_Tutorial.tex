%====================================================================%
%                                                                    %
% Copyright 2009 Mikael Granvik, Jenni Virtanen, Karri Muinonen,     %
%                Teemu Laakso, Dagmara Oszkiewicz                    %
%                                                                    %
% This file is part of OpenOrb.                                      %
%                                                                    %
% OpenOrb is free software: you can redistribute it and/or modify it %
% under the terms of the GNU General Public License as published by  %
% the Free Software Foundation, either version 3 of the License, or  %
% (at your option) any later version.                                %
%                                                                    %
% OpenOrb is distributed in the hope that it will be useful, but     %
% WITHOUT ANY WARRANTY; without even the implied warranty of         %
% MERCHANTABILITY or FITNESS FOR A PARTICULAR PURPOSE.  See the GNU  %
% General Public License for more details.                           %
%                                                                    %
% You should have received a copy of the GNU General Public License  %
% along with OpenOrb. If not, see <http://www.gnu.org/licenses/>.    %
%                                                                    %
%====================================================================%
\documentclass[12pt,english,twoside,a4paper]{report}
\usepackage{babel}
\usepackage{t1enc}
\usepackage[T1]{fontenc}
\usepackage[latin1]{inputenc}
\usepackage{epsfig}
\usepackage{amsmath}
\usepackage{amssymb}
\usepackage{graphics}
\usepackage{graphicx}
\usepackage{lscape}
\usepackage{rotating}
\usepackage{verbatim}
\usepackage{latexsym}
\usepackage{ae}
\usepackage{aecompl}
\usepackage{acronym}
\usepackage{harvard}
\usepackage{ae}    
\usepackage{aecompl}  
\pagestyle{plain} 
\renewcommand{\baselinestretch}{1}
\renewcommand{\deg}{^\circ}
\newcommand{\arcmin}{^{\prime}}
\newcommand{\arcsec}{^{\prime\prime}}
\newcommand{\h}{^\mathrm{h}}
\newcommand{\m}{^\mathrm{m}}
\newcommand{\s}{^\mathrm{s}}
\newcommand{\Mpc}{\,\mathrm{Mpc}}
\newcommand{\kpc}{\,\mathrm{kpc}}
\newcommand{\pc}{\,\mathrm{pc}}
\newcommand{\AU}{\,\mathrm{AU}}
\newcommand{\km}{\,\mathrm{km}}
\newcommand{\meter}{\,\mathrm{m}}
\newcommand{\cm}{\,\mathrm{cm}}
\newcommand{\mm}{\,\mathrm{mm}}
\newcommand{\um}{\,\mu \mathrm{m}}
\newcommand{\nm}{\,\mathrm{nm}}
\newcommand{\rad}{\,\mathrm{rad}}
\newcommand{\anno}{\,\mathrm{a}}
\newcommand{\days}{\,\mathrm{d}}
\newcommand{\vrk}{\,\mathrm{vrk}}
\newcommand{\hours}{\,\mathrm{hours}}
\newcommand{\hour}{\,\mathrm{h}}
\newcommand{\minute}{\,\mathrm{min}}
\newcommand{\second}{\,\mathrm{s}}
\newcommand{\magnitude}{^\mathrm{m}}
\newcommand{\K}{\,\mathrm{K}}
\newcommand{\TJ}{\,\mathrm{TJ}}
\newcommand{\J}{\,\mathrm{J}}
\newcommand{\N}{\,\mathrm{N}}
\newcommand{\kg}{\,\mathrm{kg}}
\newcommand{\g}{\,\mathrm{g}}
\newcommand{\AMU}{\,\mathrm{AMU}}
\newcommand{\W}{\,\mathrm{W}}
\newcommand{\MW}{\,\mathrm{MW}}
%\newcommand{\elements}{\mathbf{P}}
\newcommand{\ud}{\mathrm{d}}
%\renewcommand{\r}{\mathbf{r}}
%\renewcommand{\v}{\mathbf{v}}
\newcommand{\f}{\mathnormal{f}}
\renewcommand{\g}{\mathnormal{g}}
\title{OpenOrb Tutorial v.0.2}
\author{Mikael Granvik (mgranvik@iki.fi) \\ Jenni Virtanen (jenni.virtanen@fgi.fi)}
\date{February 19, 2009}
%\usepackage[pdftex]{hyperref} 
\begin{document}
\maketitle
\citationstyle{agsm}
\citationmode{abbr}

\pagenumbering{roman}
\setcounter{page}{1}
%\pagestyle{fancy}

\begin{verbatim}
#====================================================================#
#                                                                    #
# Copyright 2009 Mikael Granvik, Jenni Virtanen, Karri Muinonen      #
#                Teemu Laakso, Dagmara Oszkiewicz                    #
#                                                                    #
# This file is part of OpenOrb.                                      #
#                                                                    #
# OpenOrb is free software: you can redistribute it and/or modify it #
# under the terms of the GNU General Public License as published by  #
# the Free Software Foundation, either version 3 of the License, or  #
# (at your option) any later version.                                #
#                                                                    #
# OpenOrb is distributed in the hope that it will be useful, but     #
# WITHOUT ANY WARRANTY; without even the implied warranty of         #
# MERCHANTABILITY or FITNESS FOR A PARTICULAR PURPOSE.  See the GNU  #
# General Public License for more details.                           #
#                                                                    #
# You should have received a copy of the GNU General Public License  #
# along with OpenOrb. If not, see <http://www.gnu.org/licenses/>.    #
#                                                                    #
#====================================================================#
\end{verbatim}

\pagebreak

\tableofcontents

\pagebreak
\pagenumbering{arabic}
\setcounter{page}{1}

\chapter{Short introduction to OpenOrb}

\section{Theory}

In the Bayesian framework, the parameters to be estimated are treated
as random variables and the complete solution to the inverse problem
is contained in the parameters' posterior probability densities. The
posterior distribution $p_\mathrm{p}$ is proportional to the a priori
($p_\mathrm{pr}$) and the observational error ($p_\mathrm{\epsilon}$)
PDFs:
\begin{equation}
p_\mathrm{p}(\mathbf{P}) \propto p_\mathrm{pr}(\mathbf{P}) p_\mathrm{\epsilon}(\Delta
\psi (\mathbf{P}))
\end{equation}
where $\mathbf{P}$ refers to the orbital elements and $\Delta \psi
(\mathbf{P})$ stands for the observed minus computed ($O-C$) residuals
\cite{mui1993a,vir2005c}. The a priori often used is Jeffreys' a
priori which secures the invariance of the results. For example, the
collision probability does not depend on the type of orbital elements
used in the analysis \cite{vir2006a}.

If you want to transform the computed orbital-element PDF to
another set of variables such as other types of orbital elements or
ephemerides, you must also transform (that is, propagate) the weights
to the new variables \cite{mui1993a}:
\begin{equation}
p(\mathbf{F}) = \int dP p(\mathbf{P})\delta_\mathrm{D}
(\mathbf{F}-\mathbf{F}(\mathbf{P})) = \frac{1}{\det(\frac{\delta
    \mathbf{F}}{\delta \mathbf{P}})_\mathbf{F}}
p(\mathbf{P}(\mathbf{F})),
\end{equation}
where
$\mathbf{F}(\mathbf{P})=(F_1(\mathbf{P}),\ldots,F_k(\mathbf{P}))^T$ is
a set of functions of the orbital elements, $\delta_\mathrm{D}$ is
Dirac's function, and $\det(\frac{\delta \mathbf{F}}{\delta
  \mathbf{P}})_F$ is the determinant of the Jacobian matrix for the
transformation $\mathbf{P} \rightarrow \mathbf{F}$. The propagation of
the weights is done automatically for the PDFs produced by the
``official'' OpenOrb executable, \verb|oorb|.



\section{Numerical Methods} 

\subsection{Orbital inversion}

\subsubsection{Ranging}

Ranging maps the non-Gaussian orbital-element PDF with a given
number of sample orbits \cite{vir2001a,mui2001a}. Each sample orbit is
computed using the following scheme: Two observations are chosen from
the data set and a random deviate is added to all four coordinates to
mimic observational noise. Next, a random topocentric distance is
generated for the first observation date using a, typically, broad
interval. The topocentric distance for the second observation date is
generated from an interval relative to the topocentric distance on the
first observation date. Since the location of the observatory with
respect to the Sun is usually known, the four plane-of-sky coordinates
and the two topocentric distances can be transformed into two
heliocentric positions corresponding to the two observation
dates. Using well-established methods in celestial mechanics, an orbit
can be computed using the two heliocentric positions. The generated
sample orbit is then used to compute ephemerides for the other
observation dates. If the residuals are acceptable and the
PDF value is good enough with respect to the until-then best-fit
orbit, the sample orbit is accepted. The inversion can be sped up by
iteratively adjusting the intervals for the topocentric distance
either by performing the simulation for a smaller number of sample
orbits before performing the full-scale inversion \cite{vir2003a} or
by starting with only two observations and adding more observations
step by step \cite{gra2005c}.

\subsubsection{Least-squares with linearized covariances}

Least-squares with linearized covariances (LSL)\ldots

\subsection{Integration}

- Bulirsch-Stoer extrapolation method

- $n$-body dynamical model contains 8 planets + Moon + Pluto

- leading relativistic term by the Sun (perihelion shift)

\chapter{Installing OpenOrb}

\section{Software requirements}

In addition to basic Unix tools such as \verb|tar| and \verb|make|,
using OpenOrb only requires a Fortran 90/95 compiler---we are mostly
using the free \verb|gfortran| compiler (http://gcc.gnu.org/)---and
\verb|gnuplot| (http://www.gnuplot.info) for automatic generation of
plots.

\section{Producing the executable}

Extracting the source code archive will produce a directory called
\verb|OpenOrb|: \\ \\
\verb|tar xvzf OpenOrb-vN.N.tar.gz|\\ \\
Change directory to the \verb|OpenOrb| directory:\\ \\
\verb|cd OpenOrb|\\ \\
Run configuration script to set up the chosen compiler and its
command-line switches:\\ \\
\verb+./configure [COMPILER] [opt | deb]+\\ \\
Here \verb|COMPILER| can be one of the following: gfortran, g95,
lahey, intel, sun, compaq, or absoft. The option \verb|opt| will
produce optimized code for production use, whereas \verb|deb| will
produce code suitable for debugging. The compiler commands and
switches are explicitly given in (and should be modified through) the
\verb|make.config| file. Next change to the \verb|main| directory and
produce the \verb|oorb| executable: \\ \\
\verb|cd main| \\
\verb|make oorb| \\ \\
Assuming everything compiled without problems\footnote{The only known
  (compatibility) issue can be found at the end of
  \mbox{modules/cl\_options.f90} and requires commenting certain lines
  in/out. If you are using the latest gfortran compiler, everything
  should compile without errors.} the \verb|oorb| executable is now in
the \verb|main| directory.

To allow a flexible usage of the executable, you may want to make the
following changes or additions to your shell's configuration file: \\ \\ 
\verb|export PATH=$PATH:/path/to/oorb| \\ \\
If you have access to \verb|gnuplot| and want to make automatic plots
of the orbital-element PDFs, you should define the
\verb|$OORB_GNUPLOT_SCRIPTS_DIR| environment variable: \\ \\ 
\verb|export OORB_GNUPLOT_SCRIPTS_DIR=/path/to/OpenOrb/gnuplot/|


\section{Getting and processing required data files}

Access to the following data files is required to run \verb|oorb|: 
\begin{itemize}
  \item OBSCODE.dat \href{http://www.cfa.harvard.edu/iau/lists/ObsCodes.html}{http://www.cfa.harvard.edu/iau/lists/ObsCodes.html}
  \item TAI-UTC.dat \href{http://hpiers.obspm.fr/eop-pc/}{http://hpiers.obspm.fr/eop-pc/}
  \item ET-UT.dat \href{}{}
  \item de405.dat
\end{itemize}
The most recent versions of the three first files are incorporated in
the OpenOrb package. OBSCODE.dat is updated every night by the Minor
Planet Center. However, if you do not use observations from
observatories that have recently acquired an observatory code, there
is no need to update this file more often than, say, once every year
or so. Note that the format of the OBSCODE.dat is identical to the
HTML page with the exception that all HTML tags have been
deleted. TAI-UTC.dat and ET-UT.dat have to be updated manually. A
suitable interval to check the availability of updates is every six
months or so.

To generate the JPL planetary ephemeris file \verb|de405.dat|, go to
the \\ \verb|OpenOrb/data/JPL_ephemeris/| directory \\ \\ 
\verb|cd ../data/JPL_ephemeris/| \\ \\ 
and issue the following \verb|make| command: \\ \\ 
\verb|make de405| \\ \\
If successful, the ASCII versions of the de405 ephemerides will first
be automatically downloaded from the JPL FTP server, and then built to
the binary file \verb|de405.dat|. Finally, \verb|de405.dat| is copied
to the \verb|data| directory and the intermediate files are
deleted. Note that you do not have to rebuild the \verb|de405.dat|
file every time you get a new version of OpenOrb. However, since
\verb|de405.dat| is a binary file, it may have to be built separately
for every operating system. The correctness of the generated ephemeris
file can be checked with \\ \\
\verb|make de405_test| \\ \\
Finally, the \verb|OORB_DATA| environment variable, which contains the
path to the data directory (usually \verb|OpenOrb/data/|), should be
added to the environment: \\ \\
\verb|export OORB_DATA=/path/to/data/directory/| \\ \\
Again, consider adding the \verb|OORB_DATA| variable to your shell's
configuration file. 


\section{Configuration file}

The explanations and possible values for the configuration parameters
are given in Table \ref{table:conf}. The path to the configuration
file \verb|FILE| (path to default file is
\verb|OpenOrb/main/oorb.conf|) can either be given using the command
line option \verb|--conf=FILE| or using an environment variable:
\\ \\ \verb|export OORB_CONF=/path/to/configuration/FILE| \\ \\ If
neither of these are used, it is assumed that the configuration file
is called \verb|oorb.conf| and resides in the working directory. Note
that specifying the path as a command-line parameter
(\verb|--conf=FILE|) overrides the environment variable
(\verb|$OORB_CONF|) which in turn overrides the default path
(\verb|./oorb.conf|).

\begin{table}\label{table:conf}
\centering
\begin{tabular}{cc}
\hline
Option & Explanation \\
\hline
MANY OPTIONS & NOT MUCH TIME \\
\hline
\end{tabular}
\caption{See configuration file. The options should be self-explanatory\ldots}  
\end{table}



\section{Sample configuration file}

If you are using Bash, it makes life quite a bit easier if the
\verb|.bashrc| file contains the following entries:
\begin{verbatim}
export PATH=$PATH:/path/to/oorb
export OORB_CONF=/path/to/oorb/configuration/file
export OORB_DATA=/path/to/data/directory/
export OORB_GNUPLOT_SCRIPTS_DIR=/path/to/OpenOrb/gnuplot/
\end{verbatim}



\chapter{Using OpenOrb}

The basic options (input, output, etc.) are given as command-line
parameters when running \verb|oorb|, but more detailed configuration
parameters are defined using a configuration file.

\section{File formats}

\subsection{Observations}

\begin{itemize}

\item Current MPC format (.mpc).
\item New MPC format (.mpc3).

\end{itemize}

In the future, OpenOrb will be able to use the format which Pan-STARRS
will be using.

\subsection{Orbits}

Current OpenOrb format (.orb).

\subsection{Ephemeris}

Current OpenOrb format (.eph).



\section{Computing orbital-elements based on observed positions}

The \verb|oorb| executable currently has to ways of obtaining the
orbital-element PDF based on the observed positions, Ranging
(\verb|--task=ranging|) and LSL (\verb|--task=lsl|). Whereas Ranging
maps the PDF rigorously, that is, without prior
assumptions of its shape, LSL assumes that the PDF is Gaussian and
derives a so-called single-point estimate for the orbital-element
PDF. In practice, the single-point estimate consists of the nominal
orbit and the corresponding hyperellipsoid which describes the
uncertainty of the orbit. In general, Ranging is optimized for
situations with scarce data. LSL, on the other hand, should only be
used when there is enough data available to assume that the {\it true}
shape of the orbital-element PDF is indeed more or less
Gaussian. The orbital inversion is performed by using either one of
the followong commands: \\ \\
\verb|oorb --task=ranging --obs-in=FILE.mpc [ --orb-out=OUTFILE ]| \\
\verb|   [ --orb-out-separately ]| \\ \\ 
\verb|oorb --task=lsl --obs-in=FILE.mpc --orb-in=INFILE|\\
\verb|   [ --orb-out=OUTFILE ] [ --orb-out-separately ]| \\ \\ 
Here \verb|FILE.mpc| contains the observed positions in the Minor
Planet Center (MPC) format, and the file may contain more than one
object in which case the number or temporary designation separates
different sets. The observational uncertainty is specified in the
configuration file. Currently, OpenOrb also allows observations to be
fed using the ``New MPC Format'' the fate of which is still
uncertain. In the New MPC Format (OpenOrb recognizes it by the suffix
\verb|.mpc3|) observations are, among other things, allowed to have
individual uncertainties. For LSL, \verb|oorb| needs at least one
starting point which is supplied using INFILE. If
\verb|--orb-out=OUTFILE| is omitted, the resulting orbital-element
PDF is written to standard out. With the
\verb|--orb-out-separately| option the orbit or orbits for each
separate observation set is written to separate file the name of which
is defined by the designation or number of the observation set.

\subsection{Tips for using Ranging}

\begin{itemize}

\item Make a reasonable assumption for the observational uncertainty
  $\sigma_\mathrm{obs}$, e.g., the observatory's historical RMS for
  $O-C$ residuals.

\item If using relative weights for sample orbits (non-uniform
  sampling), make sure the acceptance window ($[-c_\mathrm{acc} \times
    \sigma_\mathrm{obs},+c_\mathrm{acc} \times \sigma_\mathrm{obs}]$,
  where $c_\mathrm{acc}$ is the acceptance-window multiplier) is large
  enough to allow sampling without cut-offs due to residuals. In
  practice, make sure the residual stamps figure do not show cut-offs.

\item The generation window ($[-c_\mathrm{gen} \times
  \sigma_\mathrm{obs},+c_\mathrm{gen} \times \sigma_\mathrm{obs}]$,
  where $c_\mathrm{gen}$ is the generation-window multiplier) should
  not be larger than the acceptance window. Typically, they have the
  same size, that is, $c_\mathrm{acc}=c_\mathrm{gen}$.

\item Use a number of sample orbits which is enough to do statistics,
  but is still not overkill considering running time. Typically,
  2,000--10,000 sample orbits is enough.

\item As rule of thumb, use the two-body approximation for
  Ranging. The $n$-body corrections are usually only required for
  special cases such as collision-probability estimation.

\end{itemize}

\section{Propagation of orbital elements from one epoch to another}

The orbital-element PDF can be propagated to another epoch using
the command \\ \\
\verb|oorb --task=propagation --epoch-mjd-tt=MJD --orb-in=INFILE| \\
\verb|   [ --orb-out=OUTFILE ]| \\ \\
The mode of propagation, that is, analytical two-body propagation or
numerical $n$-body propagation using an integrator, can be defined in
the configuration file. If \verb|--orb-out=OUTFILE| is omitted, the
resulting orbital-element PDF is written to standard out.

\section{Computing ephemerides}

Topocentric ephemerides and their uncertainties are computed based on
the orbital-element PDF using the command \\ \\
\verb+oorb --task=ephemeris --obs-code=CODE [ --epoch-mjd-utc=MJD |+\\
\verb+     --epoch-mjd-tt=MJD | --timespan=DT1 --step=DT2 ]+\\
\verb+     --orb-in=INFILE+\\ \\
where \verb|CODE| is the observatory code assigned by MPC and MJD is
the Modified Julian Date of the desired ephemerides either in UTC or
TT. DT1 is the timespan (in days) over which ephemerides should be
computed when starting from the epoch of the orbital-elements and DT2
is the frequency (in days) of the ephemerides. Positive values for DT1
and DT2 indicate going forwards in time from the epoch, whereas
negative values indicate going backwards. The ephemeris prediction
based on the sampled PDF is a discrete set of predicted positions
on the sky and their individual weights, and the ephemeris prediction
based on the single-point estimate is the nominal position with a
3-$\sigma$ uncertainty ellipse. Note that the non-Gaussian uncertainty
region equivalent to the Gaussian 3-$\sigma$ region can also be
obtained for the sampled prediction by normalizing the sum of the
weights to unity and requiring that 99.73002\% of the total weight
(probability mass) is within the boundaries. The ephemerides are
written to standard out.

\section{Dynamical classification}

%Dynamical classification without the use of a priori knowledge is
%currently included in the \verb|*.sor| file. The file is produced for
%every observation set which is run through Ranging.
For cases where the orbital-element PDF has been sampled, it is
possible to classify the object in a dynamical sense by using the
relative weight of sample orbits. The classification
\\ \\ \verb|oorb --task=classification --orb-in=INFILE| \\ \\ The
results are written to standard output.  The classification criteria
can be changed by modifying the \verb|getGroupProbabilities|
subroutine in \\ \verb|StochastiOrbit_class.f90|. Note that the
current implementation of the dynamical classification is only done in
the osculating $(a,e,i,q,Q)$ space, which means that, e.g., the
classification of Jupiter Trojans is not carried out rigorously.

\section{Conversion between file types}

Use \\ \\
\verb|oorb --task=mpctompc3 --obs-in=FILE1 [ --obs-out=FILE2]| \\ \\
to convert an observation file (FILE1) formatted according to the
current MPC format to a file (FILE2) formatted according to the
proposed new MPC format. Observational uncertainties are taken from
the configuration file. If they are not defined, the default
uncertainty is zero. Use \\ \\
\verb|oorb --task=mpc3tompc --obs-in=FILE1 [ --obs-out=FILE2]| \\ \\
to convert an observation file (FILE1) formatted according to the
proposed new MPC format to a file (FILE2) formatted according to the
current MPC format. Note that information regarding observational
uncertainties is lost during the conversion. Use \\ \\
\verb|oorb --task=astorbtoorb --astorb=FILE1 [ --orb-out=FILE2]| \\ \\
to convert the orbital elements and the (H,G) values in Ted Bowell's
astorb.dat file to the .orb format. Here FILE1 is the astorb.dat file
and FILE2 is the output .orb file. Again, if FILE2 is omitted the
output is sent to standard out.

\section{Example}

In the \verb|OpenOrb/test/| directory you can find two almost
identical observation files in the MPC format,
\verb|K08K42V_ranging.mpc| and \verb|K08K42V_lsl.mpc|. The difference
between these files is that in the previous file all but the three
first observations have been commented out, whereas in the latter none
are commented out.

First, compute a Ranging solution for observation set \#1:
\\ \\ 
\verb|oorb --task=ranging --obs-in=K08K42V_ranging.mpc \| \\
\verb|     --orb-out=K08K42V_ranging.orb|
\\ \\
The output is written both in the form of an orbital-element
PDF file called \verb|K08K42V_ranging.orb| and a more general
output file called \\ \verb|K08K42V_ranging.sor|. Moreover, two separate
plots were also produced:
\verb|K08K0042V_0.08_sor_keplerian_results.eps.gz| shows the
orbital-element PDF and
\verb|K08K0042V_0.08_sor_residual_stamps.eps.gz| shows stamps of the
$O-C$ residual distributions corresponding to each observation
included in the inversion.

Then, use the orbits produced by Ranging to initiate the differential
correction (LSL) using observation set \#2: \\ \\ 
\verb|oorb --task=lsl --obs-in=K08K42V_lsl.mpc \| \\
\verb|     --orb-in=K08K42V_ranging.orb --orb-out=K08K42V_lsl.orb| \\ \\ 
Again, the output is written in the form of an orbital-element file
called \verb|K08K42V_lsl.orb| which contains the nominal orbit and the
standard deviations and correlations for the elements, and a general
output file called \verb|K08K42V_lsl.ls|. The nominal orbit and the
1-$\sigma$ and the 3-$\sigma$ uncertainty ellipsoids are shown in plot
\verb|K08K0042V_37.80_ls_keplerian_results.eps.gz|.

To make a dynamical classification of the object, do \\ \\
\verb|oorb --task=classification --orb-in=K08K42V_ranging.orb| \\ \\
The result strongly suggests that the object is something unknown to
the current classification scheme and, in fact, the object is the
first retrograde TNO discovered.

To propagate the orbital-element PDF to the current date
(2008-08-12) epoch, do \\ \\ 
\verb|oorb --task=propagation --orb-in=K08K42V_ranging.orb \| \\ 
\verb|     --epoch-mjd-tt=54690 --orb-out=K08K42V_ranging_54690.orb| \\ \\ 
The file \verb|K08K42V_ranging_54690.orb| contains the orbital-element
PDF at the new epoch.

To compute follow-up ephemerides (topocentric distance, R.A., Dec.,
and their corresponding time derivatives) for 10 dates with one-day
intervals starting at 2008-08-12.0 TT, use \\ \\
\verb|oorb --task=ephemeris --orb-in=K08K42V_ranging_54690.orb \| \\ 
\verb|     --timespan=10.0 --step=1.0| \\ \\ 
The ephemerides are written to standard out.

\chapter{Developing your own applications based on OpenOrb}

It is reasonably straightforward to develop your own applications in
the OpenOrb framework due to the object-oriented approach. Although
Fortran 90/95 is not a pure object-oriented language, we have mimicked
object-oriented concepts in the software by using the guidelines by
\citeasnoun{dec1997a}.

More to be added later if there is interest\ldots


\chapter{Communication}

\section{Getting help}

Due to manpower limitations, we cannot promise to give personal
assistance in all situations. It is assumed that the user is familiar
with state-of-the-art orbit computation methods and Bayesian inversion
theory through published literature
\citeaffixed{vir2005c,bow2003a,vir2008a}{for reviews, see}. However,
if you intend to publish results obtained with OpenOrb and have doubt
regarding the interpretation of the results, we encourage you to
contact either Mikael Granvik (mgranvik@iki.fi) or Jenni Virtanen
(jenni.virtanen@fgi.fi) for assistance.

If needed, there may be help available for developing source code
based on OpenOrb to solve more complicated orbit-computation-related
problems. However, the help has a price, which is typically to be paid
in the currency which our funding agencies understand and accept, that
is, as a co-authorship in the resulting publication(s).

\section{Reporting bugs}

OpenOrb has been tested during the years by comparing the results to
the results obtained with either our older software or by software by
colleagues. If you nevertheless find a bug, please report it to us so
that we can fix the problem. The report should be concise and contain
everything needed for reproducing the erroneous result. Send the
report to MG (mgranvik@iki.fi) with the text ``OpenOrb bug'' in the
subject line.

\section{Acknowledging the use of OpenOrb}

If you find OpenOrb useful in your work, please acknowledge our
efforts by citing, in addition to the relevant original papers, the
following abstract (which will hopefully evolve into an ACM 2008
conference proceedings ma\-nu\-script) on the OpenOrb software package: \\ \\ 

\noindent Granvik, M., Virtanen, J., and Muinonen, K. (2008),
`OpenOrb: Open-source asteroid-orbit-computation software including
statistical orbital ranging', Asteroids, Comets, Meteors 2008,
Abstract \#8206, CD-ROM, LPI Contribution No.\ 1405.



\bibliographystyle{agsm}
\bibliography{asteroid,educational,fortran}
%\nocite{mui1999a}

\end{document}
